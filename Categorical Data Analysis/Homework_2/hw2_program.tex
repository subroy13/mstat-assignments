\documentclass[]{article}
\usepackage{lmodern}
\usepackage{amssymb,amsmath}
\usepackage{ifxetex,ifluatex}
\usepackage{fixltx2e} % provides \textsubscript
\ifnum 0\ifxetex 1\fi\ifluatex 1\fi=0 % if pdftex
  \usepackage[T1]{fontenc}
  \usepackage[utf8]{inputenc}
\else % if luatex or xelatex
  \ifxetex
    \usepackage{mathspec}
  \else
    \usepackage{fontspec}
  \fi
  \defaultfontfeatures{Ligatures=TeX,Scale=MatchLowercase}
\fi
% use upquote if available, for straight quotes in verbatim environments
\IfFileExists{upquote.sty}{\usepackage{upquote}}{}
% use microtype if available
\IfFileExists{microtype.sty}{%
\usepackage{microtype}
\UseMicrotypeSet[protrusion]{basicmath} % disable protrusion for tt fonts
}{}
\usepackage[margin=1in]{geometry}
\usepackage{hyperref}
\hypersetup{unicode=true,
            pdftitle={HW2 Program},
            pdfauthor={Subhrajyoty Roy},
            pdfborder={0 0 0},
            breaklinks=true}
\urlstyle{same}  % don't use monospace font for urls
\usepackage{color}
\usepackage{fancyvrb}
\newcommand{\VerbBar}{|}
\newcommand{\VERB}{\Verb[commandchars=\\\{\}]}
\DefineVerbatimEnvironment{Highlighting}{Verbatim}{commandchars=\\\{\}}
% Add ',fontsize=\small' for more characters per line
\usepackage{framed}
\definecolor{shadecolor}{RGB}{248,248,248}
\newenvironment{Shaded}{\begin{snugshade}}{\end{snugshade}}
\newcommand{\KeywordTok}[1]{\textcolor[rgb]{0.13,0.29,0.53}{\textbf{#1}}}
\newcommand{\DataTypeTok}[1]{\textcolor[rgb]{0.13,0.29,0.53}{#1}}
\newcommand{\DecValTok}[1]{\textcolor[rgb]{0.00,0.00,0.81}{#1}}
\newcommand{\BaseNTok}[1]{\textcolor[rgb]{0.00,0.00,0.81}{#1}}
\newcommand{\FloatTok}[1]{\textcolor[rgb]{0.00,0.00,0.81}{#1}}
\newcommand{\ConstantTok}[1]{\textcolor[rgb]{0.00,0.00,0.00}{#1}}
\newcommand{\CharTok}[1]{\textcolor[rgb]{0.31,0.60,0.02}{#1}}
\newcommand{\SpecialCharTok}[1]{\textcolor[rgb]{0.00,0.00,0.00}{#1}}
\newcommand{\StringTok}[1]{\textcolor[rgb]{0.31,0.60,0.02}{#1}}
\newcommand{\VerbatimStringTok}[1]{\textcolor[rgb]{0.31,0.60,0.02}{#1}}
\newcommand{\SpecialStringTok}[1]{\textcolor[rgb]{0.31,0.60,0.02}{#1}}
\newcommand{\ImportTok}[1]{#1}
\newcommand{\CommentTok}[1]{\textcolor[rgb]{0.56,0.35,0.01}{\textit{#1}}}
\newcommand{\DocumentationTok}[1]{\textcolor[rgb]{0.56,0.35,0.01}{\textbf{\textit{#1}}}}
\newcommand{\AnnotationTok}[1]{\textcolor[rgb]{0.56,0.35,0.01}{\textbf{\textit{#1}}}}
\newcommand{\CommentVarTok}[1]{\textcolor[rgb]{0.56,0.35,0.01}{\textbf{\textit{#1}}}}
\newcommand{\OtherTok}[1]{\textcolor[rgb]{0.56,0.35,0.01}{#1}}
\newcommand{\FunctionTok}[1]{\textcolor[rgb]{0.00,0.00,0.00}{#1}}
\newcommand{\VariableTok}[1]{\textcolor[rgb]{0.00,0.00,0.00}{#1}}
\newcommand{\ControlFlowTok}[1]{\textcolor[rgb]{0.13,0.29,0.53}{\textbf{#1}}}
\newcommand{\OperatorTok}[1]{\textcolor[rgb]{0.81,0.36,0.00}{\textbf{#1}}}
\newcommand{\BuiltInTok}[1]{#1}
\newcommand{\ExtensionTok}[1]{#1}
\newcommand{\PreprocessorTok}[1]{\textcolor[rgb]{0.56,0.35,0.01}{\textit{#1}}}
\newcommand{\AttributeTok}[1]{\textcolor[rgb]{0.77,0.63,0.00}{#1}}
\newcommand{\RegionMarkerTok}[1]{#1}
\newcommand{\InformationTok}[1]{\textcolor[rgb]{0.56,0.35,0.01}{\textbf{\textit{#1}}}}
\newcommand{\WarningTok}[1]{\textcolor[rgb]{0.56,0.35,0.01}{\textbf{\textit{#1}}}}
\newcommand{\AlertTok}[1]{\textcolor[rgb]{0.94,0.16,0.16}{#1}}
\newcommand{\ErrorTok}[1]{\textcolor[rgb]{0.64,0.00,0.00}{\textbf{#1}}}
\newcommand{\NormalTok}[1]{#1}
\usepackage{graphicx,grffile}
\makeatletter
\def\maxwidth{\ifdim\Gin@nat@width>\linewidth\linewidth\else\Gin@nat@width\fi}
\def\maxheight{\ifdim\Gin@nat@height>\textheight\textheight\else\Gin@nat@height\fi}
\makeatother
% Scale images if necessary, so that they will not overflow the page
% margins by default, and it is still possible to overwrite the defaults
% using explicit options in \includegraphics[width, height, ...]{}
\setkeys{Gin}{width=\maxwidth,height=\maxheight,keepaspectratio}
\IfFileExists{parskip.sty}{%
\usepackage{parskip}
}{% else
\setlength{\parindent}{0pt}
\setlength{\parskip}{6pt plus 2pt minus 1pt}
}
\setlength{\emergencystretch}{3em}  % prevent overfull lines
\providecommand{\tightlist}{%
  \setlength{\itemsep}{0pt}\setlength{\parskip}{0pt}}
\setcounter{secnumdepth}{0}
% Redefines (sub)paragraphs to behave more like sections
\ifx\paragraph\undefined\else
\let\oldparagraph\paragraph
\renewcommand{\paragraph}[1]{\oldparagraph{#1}\mbox{}}
\fi
\ifx\subparagraph\undefined\else
\let\oldsubparagraph\subparagraph
\renewcommand{\subparagraph}[1]{\oldsubparagraph{#1}\mbox{}}
\fi

%%% Use protect on footnotes to avoid problems with footnotes in titles
\let\rmarkdownfootnote\footnote%
\def\footnote{\protect\rmarkdownfootnote}

%%% Change title format to be more compact
\usepackage{titling}

% Create subtitle command for use in maketitle
\newcommand{\subtitle}[1]{
  \posttitle{
    \begin{center}\large#1\end{center}
    }
}

\setlength{\droptitle}{-2em}

  \title{HW2 Program}
    \pretitle{\vspace{\droptitle}\centering\huge}
  \posttitle{\par}
    \author{Subhrajyoty Roy}
    \preauthor{\centering\large\emph}
  \postauthor{\par}
      \predate{\centering\large\emph}
  \postdate{\par}
    \date{August 25, 2019}


\begin{document}
\maketitle

We first load the required packages in \emph{R}.

\begin{Shaded}
\begin{Highlighting}[]
\KeywordTok{library}\NormalTok{(vcd)}
\end{Highlighting}
\end{Shaded}

\begin{verbatim}
Loading required package: grid
\end{verbatim}

\begin{Shaded}
\begin{Highlighting}[]
\KeywordTok{library}\NormalTok{(lpSolve)}
\end{Highlighting}
\end{Shaded}

Now, we load the \emph{SexualFun} data from \textbf{vcd} package.

\begin{Shaded}
\begin{Highlighting}[]
\NormalTok{data <-}\StringTok{ }\NormalTok{vcd}\OperatorTok{::}\NormalTok{SexualFun}
\KeywordTok{ftable}\NormalTok{(data)}
\end{Highlighting}
\end{Shaded}

\begin{verbatim}
             Wife Never Fun Fairly Often Very Often Always fun
Husband                                                       
Never Fun                 7            7          2          3
Fairly Often              2            8          3          7
Very Often                1            5          4          9
Always fun                2            8          9         14
\end{verbatim}

We first obtain the actual values of Cohen's Kappa \(\kappa\) and the
measure \(\lambda_r\) for the actual table;

\begin{Shaded}
\begin{Highlighting}[]
\NormalTok{kp <-}\StringTok{ }\KeywordTok{Kappa}\NormalTok{(data)   }\CommentTok{#compute the actual Kappa }
\KeywordTok{print}\NormalTok{(kp)}
\end{Highlighting}
\end{Shaded}

\begin{verbatim}
            value     ASE     z Pr(>|z|)
Unweighted 0.1293 0.06860 1.885 0.059387
Weighted   0.2374 0.07832 3.031 0.002437
\end{verbatim}

We note that the actual value of Cohen's Kappa for \emph{SexualFun} data
is \(0.12933025\).

Consider the following code which computes \(\lambda_r\);

\begin{Shaded}
\begin{Highlighting}[]
\NormalTok{n <-}\StringTok{ }\KeywordTok{sum}\NormalTok{(data)}
\NormalTok{rowMar <-}\StringTok{ }\KeywordTok{rowSums}\NormalTok{(data)}\OperatorTok{/}\NormalTok{n}
\NormalTok{colMar <-}\StringTok{ }\KeywordTok{colSums}\NormalTok{(data)}\OperatorTok{/}\NormalTok{n}

\NormalTok{a <-}\StringTok{ }\KeywordTok{sum}\NormalTok{(}\KeywordTok{diag}\NormalTok{(data))}\OperatorTok{/}\NormalTok{n}
\NormalTok{b <-}\StringTok{ }\KeywordTok{max}\NormalTok{((rowMar }\OperatorTok{+}\StringTok{ }\NormalTok{colMar)}\OperatorTok{/}\DecValTok{2}\NormalTok{)}

\NormalTok{lambda <-}\StringTok{ }\NormalTok{(a }\OperatorTok{-}\StringTok{ }\NormalTok{b)}\OperatorTok{/}\NormalTok{(}\DecValTok{1}\OperatorTok{-}\NormalTok{b)}
\KeywordTok{print}\NormalTok{(lambda)}
\end{Highlighting}
\end{Shaded}

\begin{verbatim}
[1] 0
\end{verbatim}

We find that the actual value of \(\lambda_r\) for \emph{SexualFun} data
is \(0\).

Now, we consider the minimization and maximization problem, where we try
to find extremal values of \(\sum_i p_{ii}\) or correspondingly
\(\sum_i n_{ii}\). We shall use \textbf{lpSolve} package to solve the
corresponding linear programming for us.

\begin{Shaded}
\begin{Highlighting}[]
\NormalTok{lpMax =}\StringTok{ }\KeywordTok{lp.transport}\NormalTok{(}\DataTypeTok{cost.mat =} \KeywordTok{diag}\NormalTok{(}\DecValTok{4}\NormalTok{), }\DataTypeTok{direction =} \StringTok{"max"}\NormalTok{, }
             \DataTypeTok{row.signs =} \KeywordTok{rep}\NormalTok{(}\StringTok{"=="}\NormalTok{, }\DecValTok{4}\NormalTok{), }\DataTypeTok{row.rhs =} \KeywordTok{rowSums}\NormalTok{(data), }
             \DataTypeTok{col.signs =} \KeywordTok{rep}\NormalTok{(}\StringTok{"=="}\NormalTok{, }\DecValTok{4}\NormalTok{), }\DataTypeTok{col.rhs =} \KeywordTok{colSums}\NormalTok{(data))}
\end{Highlighting}
\end{Shaded}

The maximizing configuration would look like;

\begin{Shaded}
\begin{Highlighting}[]
\KeywordTok{print}\NormalTok{(lpMax}\OperatorTok{$}\NormalTok{solution)}
\end{Highlighting}
\end{Shaded}

\begin{verbatim}
     [,1] [,2] [,3] [,4]
[1,]   12    7    0    0
[2,]    0   20    0    0
[3,]    0    1   18    0
[4,]    0    0    0   33
\end{verbatim}

Therefore, the maximum value of Cohen's kappa and \(\lambda_r\) is
obtained using the following code;

\begin{Shaded}
\begin{Highlighting}[]
\KeywordTok{Kappa}\NormalTok{(lpMax}\OperatorTok{$}\NormalTok{solution)}
\end{Highlighting}
\end{Shaded}

\begin{verbatim}
            value     ASE     z   Pr(>|z|)
Unweighted 0.8799 0.03969 22.17 6.661e-109
Weighted   0.9291 0.02348 39.57  0.000e+00
\end{verbatim}

\begin{Shaded}
\begin{Highlighting}[]
\NormalTok{a <-}\StringTok{ }\NormalTok{lpMax}\OperatorTok{$}\NormalTok{objval}\OperatorTok{/}\NormalTok{n}
\NormalTok{lambda <-}\StringTok{ }\NormalTok{(a }\OperatorTok{-}\StringTok{ }\NormalTok{b)}\OperatorTok{/}\NormalTok{(}\DecValTok{1}\OperatorTok{-}\NormalTok{b)}
\KeywordTok{print}\NormalTok{(lambda)}
\end{Highlighting}
\end{Shaded}

\begin{verbatim}
[1] 0.862069
\end{verbatim}

Therefore, the maximum value of Cohen's Kappa given the marginals is
\(0.8799\), while the maximum value of \(\lambda_r\) given the marginals
is \(0.8621\).

We use similar method to find the minimum value of these measures given
the marginals.

\begin{Shaded}
\begin{Highlighting}[]
\NormalTok{lpMin =}\StringTok{ }\KeywordTok{lp.transport}\NormalTok{(}\DataTypeTok{cost.mat =} \KeywordTok{diag}\NormalTok{(}\DecValTok{4}\NormalTok{), }\DataTypeTok{direction =} \StringTok{"min"}\NormalTok{, }
             \DataTypeTok{row.signs =} \KeywordTok{rep}\NormalTok{(}\StringTok{"=="}\NormalTok{, }\DecValTok{4}\NormalTok{), }\DataTypeTok{row.rhs =} \KeywordTok{rowSums}\NormalTok{(data), }
             \DataTypeTok{col.signs =} \KeywordTok{rep}\NormalTok{(}\StringTok{"=="}\NormalTok{, }\DecValTok{4}\NormalTok{), }\DataTypeTok{col.rhs =} \KeywordTok{colSums}\NormalTok{(data))}
\end{Highlighting}
\end{Shaded}

The maximizing configuration would look like;

\begin{Shaded}
\begin{Highlighting}[]
\KeywordTok{print}\NormalTok{(lpMin}\OperatorTok{$}\NormalTok{solution)}
\end{Highlighting}
\end{Shaded}

\begin{verbatim}
     [,1] [,2] [,3] [,4]
[1,]    0    0    0   19
[2,]    0    0    6   14
[3,]    0   19    0    0
[4,]   12    9   12    0
\end{verbatim}

\begin{Shaded}
\begin{Highlighting}[]
\KeywordTok{Kappa}\NormalTok{(lpMin}\OperatorTok{$}\NormalTok{solution)}
\end{Highlighting}
\end{Shaded}

\begin{verbatim}
             value     ASE      z   Pr(>|z|)
Unweighted -0.3661 0.01593 -22.98 7.992e-117
Weighted   -0.5607 0.02691 -20.84  1.951e-96
\end{verbatim}

\begin{Shaded}
\begin{Highlighting}[]
\NormalTok{a <-}\StringTok{ }\NormalTok{lpMin}\OperatorTok{$}\NormalTok{objval}\OperatorTok{/}\NormalTok{n}
\NormalTok{lambda <-}\StringTok{ }\NormalTok{(a }\OperatorTok{-}\StringTok{ }\NormalTok{b)}\OperatorTok{/}\NormalTok{(}\DecValTok{1}\OperatorTok{-}\NormalTok{b)}
\KeywordTok{print}\NormalTok{(lambda)}
\end{Highlighting}
\end{Shaded}

\begin{verbatim}
[1] -0.5689655
\end{verbatim}

Therefore, the maximum value of Cohen's Kappa given the marginals is
\(-0.3661\), while the maximum value of \(\lambda_r\) given the
marginals is \(-0.5689\).


\end{document}
