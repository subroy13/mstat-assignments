\documentclass[12pt]{article}
\usepackage[usenames]{color} %used for font color
\usepackage{amsmath,amssymb,amsthm,amsfonts} %maths
\usepackage[utf8]{inputenc} %useful to type directly accentuated characters
\usepackage{hyperref}
\usepackage[margin=0.3in]{geometry}
\usepackage{graphicx}

% setting up R coding environment
\usepackage{color}
\usepackage{framed}
\definecolor{shadecolor}{RGB}{248,248,248}
\newenvironment{Shaded}{\begin{snugshade}}{\end{snugshade}}
\newcommand{\KeywordTok}[1]{\textcolor[rgb]{0.13,0.29,0.53}{\textbf{#1}}}
\newcommand{\DataTypeTok}[1]{\textcolor[rgb]{0.13,0.29,0.53}{#1}}
\newcommand{\DecValTok}[1]{\textcolor[rgb]{0.00,0.00,0.81}{#1}}
\newcommand{\BaseNTok}[1]{\textcolor[rgb]{0.00,0.00,0.81}{#1}}
\newcommand{\FloatTok}[1]{\textcolor[rgb]{0.00,0.00,0.81}{#1}}
\newcommand{\ConstantTok}[1]{\textcolor[rgb]{0.00,0.00,0.00}{#1}}
\newcommand{\CharTok}[1]{\textcolor[rgb]{0.31,0.60,0.02}{#1}}
\newcommand{\SpecialCharTok}[1]{\textcolor[rgb]{0.00,0.00,0.00}{#1}}
\newcommand{\StringTok}[1]{\textcolor[rgb]{0.31,0.60,0.02}{#1}}
\newcommand{\VerbatimStringTok}[1]{\textcolor[rgb]{0.31,0.60,0.02}{#1}}
\newcommand{\SpecialStringTok}[1]{\textcolor[rgb]{0.31,0.60,0.02}{#1}}
\newcommand{\ImportTok}[1]{#1}
\newcommand{\CommentTok}[1]{\textcolor[rgb]{0.56,0.35,0.01}{\textit{#1}}}
\newcommand{\DocumentationTok}[1]{\textcolor[rgb]{0.56,0.35,0.01}{\textbf{\textit{#1}}}}
\newcommand{\AnnotationTok}[1]{\textcolor[rgb]{0.56,0.35,0.01}{\textbf{\textit{#1}}}}
\newcommand{\CommentVarTok}[1]{\textcolor[rgb]{0.56,0.35,0.01}{\textbf{\textit{#1}}}}
\newcommand{\OtherTok}[1]{\textcolor[rgb]{0.56,0.35,0.01}{#1}}
\newcommand{\FunctionTok}[1]{\textcolor[rgb]{0.00,0.00,0.00}{#1}}
\newcommand{\VariableTok}[1]{\textcolor[rgb]{0.00,0.00,0.00}{#1}}
\newcommand{\ControlFlowTok}[1]{\textcolor[rgb]{0.13,0.29,0.53}{\textbf{#1}}}
\newcommand{\OperatorTok}[1]{\textcolor[rgb]{0.81,0.36,0.00}{\textbf{#1}}}
\newcommand{\BuiltInTok}[1]{#1}
\newcommand{\ExtensionTok}[1]{#1}
\newcommand{\PreprocessorTok}[1]{\textcolor[rgb]{0.56,0.35,0.01}{\textit{#1}}}
\newcommand{\AttributeTok}[1]{\textcolor[rgb]{0.77,0.63,0.00}{#1}}
\newcommand{\RegionMarkerTok}[1]{#1}
\newcommand{\InformationTok}[1]{\textcolor[rgb]{0.56,0.35,0.01}{\textbf{\textit{#1}}}}
\newcommand{\WarningTok}[1]{\textcolor[rgb]{0.56,0.35,0.01}{\textbf{\textit{#1}}}}
\newcommand{\AlertTok}[1]{\textcolor[rgb]{0.94,0.16,0.16}{#1}}
\newcommand{\ErrorTok}[1]{\textcolor[rgb]{0.64,0.00,0.00}{\textbf{#1}}}
\newcommand{\NormalTok}[1]{#1}


\usepackage{booktabs}
\usepackage{pgfplots}
\usepackage{pgfplotstable}

\usepackage{float}

\usepackage{enumitem}

\newcommand\independent{\protect\mathpalette{\protect\independenT}{\perp}}
\def\independenT#1#2{\mathrel{\rlap{$#1#2$}\mkern2mu{#1#2}}}

\newtheoremstyle{problemstyle}  							% <name>
        {3pt}                                               % <space above>
        {3pt}                                               % <space below>
        {\normalfont}                               		% <body font>
        {}                                                  % <indent amount}
        {\bfseries}                 						% <theorem head font>
        {\normalfont\bfseries.}         					% <punctuation after theorem head>
        {.5em}                                          	% <space after theorem head>
        {}                                                  % <theorem head spec (can be left empty, meaning `normal')>
\theoremstyle{problemstyle}

\newtheorem{problem}{}
\newtheorem{solution}{Solution}
\newtheorem*{solution*}{Solution}

\newcommand{\createcontingencytable}[4]{ %
% #1=table name
% #2=first column name
% #3=new row sum name
% #4=new column sum name
\pgfplotstablecreatecol[
    create col/assign/.code={% In each row ... 
        \def\rowsum{0}
        \pgfmathtruncatemacro\maxcolindex{\pgfplotstablecols-1}
        % ... loop over all columns, summing up the elements
        \pgfplotsforeachungrouped \col in {1,...,\maxcolindex}{
            \pgfmathsetmacro\rowsum{\rowsum+\thisrowno{\col}}
        }
        \pgfkeyslet{/pgfplots/table/create col/next content}\rowsum
    }
]{#3}{#1}%
%
% Transpose the table, so we can repeat the summation step for the columns
\pgfplotstabletranspose[colnames from={#2},input colnames to={#2}]{\intermediatetable}{#1}
%
% Sums for each column
\pgfplotstablecreatecol[
    create col/assign/.code={%
        \def\colsum{0}
        \pgfmathtruncatemacro\maxcolindex{\pgfplotstablecols-1}
        \pgfplotsforeachungrouped \col in {1,...,\maxcolindex}{
            \pgfmathsetmacro\colsum{\colsum+\thisrowno{\col}}
        }
        \pgfkeyslet{/pgfplots/table/create col/next content}\colsum
    }
]{#4}\intermediatetable
%
% Transpose back to the original form
\pgfplotstabletranspose[colnames from=#2, input colnames to=#2]{\contingencytable}{\intermediatetable}
}
%

\makeatletter
\newcommand*\bigcdot{\mathpalette\bigcdot@{.5}}
\newcommand*\bigcdot@[2]{\mathbin{\vcenter{\hbox{\scalebox{#2}{$\m@th#1\bullet$}}}}}
\makeatother

\def\ed { \stackrel{d}{=} }
\def\convd { \stackrel{d}{\rightarrow} }
\def\corr{\mathrm{corr}}
\def\normal{\mathcal{N}}
\def\R{\mathbb{R}}
\def\Pr{\mathbb{P}}
\def\M{\mathcal{M}}
\def\F{\mathcal{F}}

\newcommand{\indep}{\mathrel{\text{\scalebox{1.07}{$\perp\mkern-10mu\perp$}}}}


\begin{document}

\begin{center}{\large\textbf{Homework 2} \hfill \large \textit{Categorical Data Analysis, S. S. Mukherjee, Fall 2019}} 
\end{center}
\hrule\hrule\vskip3pt
Topics: Measures of association and agreement \hfill Due on August 29, 2019\vskip3pt
\hrule\hrule\vskip3pt\noindent
Name of student: SUBHRAJYOTY ROY \\
Roll number: MB1911
\vskip3pt\noindent	
%%%%%%%%%%%%%%%%%%%%%%%%%%%%%%%%%%%%%%%%%%%%%%%%%%
% Problem 1
%%%%%%%%%%%%%%%%%%%%%%%%%%%%%%%%%%%%%%%%%%%%%%%%%%
\begin{problem}
\textbf{Relationship between Kruskal and Goodman's $\lambda$ and Yule's $Y$} \hfill [8]\vskip3pt\noindent
% Problem statement
Consider a $2 \times 2$ multinomial contingency table. Show that the odds ratio $\theta = \frac{p_{11} p_{22}}{p_{12}p_{21}}$ is invariant under transformations of tables of the form
\[
    (p_{ij}) \mapsto (s_i t_j p_{ij}), \quad s_i, t_j > 0.
\]
Now show that one can always make the marginals equal to $1/2$ by choosing a suitable transformation of the above type (i.e. without changing the odds ratio). Show that for such a transformed table Kruskal and Goodman's $\lambda_{C\mid R} = \lambda_{R \mid C} = |Y|$, where $Y = \frac{\sqrt{\theta} - 1}{\sqrt{\theta} + 1}$ is Yule's measure of colligation.
\end{problem}
%%%%%%%%%%%%%%%%%%%%%%%%%%%%%%%%%%%%%%%%%%%%%%%%%%
% Solution to problem 1
%%%%%%%%%%%%%%%%%%%%%%%%%%%%%%%%%%%%%%%%%%%%%%%%%%
\begin{solution}
 	We consider the odds ratio under the transformation $(p_{ij}) \mapsto (s_i t_j p_{ij}), \quad s_i, t_j > 0$. Let us call $s_i t_j p_{ij} = p'_{ij}$. 
 	\begin{align*}
 		\theta_{\text{new}} & = \frac{p'_{11}p'_{22}}{p'_{12}p'_{21}}\\
 		& = \frac{s_1t_1p_{11}\times s_2t_2p_{22}}{s_1t_2p_{12}\times s_2t_1p_{21}}\\
 		& = \frac{p_{11}p_{22}}{p_{12}p_{21}} \text{, since } s_i, t_j > 0\\
 		& = \theta_{\text{old}}
 	\end{align*}
 	
 	For some choice of $s_i, t_j$, we have the following contingency table;
 	\begin{table}[H]
 		\centering 
 		\begin{tabular}{cccc}
 			\toprule
 			 & $C=1$ & $C=2$\\
 			\midrule 
 			$R=1$ & $s_1t_1p_{11}$ & $s_1t_2p_{12}$\\
 			$R=2$ & $s_2t_1p_{21}$ & $s_2t_2p_{22}$\\
 			\bottomrule
 		\end{tabular}	
 	\end{table}
 	
 	Note that, if we have all the marginals equal to $1/2$, such a contingency table would look as follows;
 	\begin{table}[H]
 		\centering 
 		\begin{tabular}{cccc}
 			\toprule
 			 & $C=1$ & $C=2$ & Total\\
 			\midrule 
 			$R = 1$ & $x$ & $\frac{1}{2} - x$ & $\frac{1}{2}$\\
 			$R=2$ & $\frac{1}{2} - x$ & $x$ & $\frac{1}{2}$\\
 			\midrule
 			Total & $\frac{1}{2}$ & $\frac{1}{2}$ & 1\\
 			\bottomrule
 		\end{tabular}	
 	\end{table}
 	for some $x \leq \frac{1}{2}$. Now, consider the following choice;
 	\begin{align}
 		\label{eqn:s1}
 		s_1 & = \frac{x}{p_{11}}\\
 		\label{eqn:t1}
 		t_1 & = 1\\
 		\label{eqn:s2}
 		s_2 & = \frac{0.5 - x}{p_{21}}\\
 		\label{eqn:t2}
 		t_2 & = \frac{(0.5-x)}{x} \times \frac{p_{11}}{p_{12}}
 	\end{align}
 	Note that, the above choices of $s_i, t_j$ transforms the entries at $(1,1), (1,2), (2,1)$ to the desired choices. However, the entry at $(2,2)$ position matches only if;
 	\begin{align*}
 		& \frac{(0.5 - x)}{p_{21}}\times \frac{(0.5-x)}{x}\frac{p_{11}}{p_{12}}\times p_{22} = x\\
 		\Rightarrow & \frac{(0.5 - x)^2}{x} \frac{p_{11}p_{22}}{p_{12}p_{21}} = x\\
 		\Rightarrow & \frac{x^2}{(0.5-x)^2} = \theta \text{, the odds ratio}\\
 		\Rightarrow & \frac{x}{0.5 -x} = \sqrt{\theta} \text{, since both sides are positive}\\
 		\Rightarrow & x = \frac{1}{2}\frac{\sqrt{\theta}}{(1 + \sqrt{\theta})}	
 	\end{align*}
 	
 	Therefore, if we choose such an $x$, we can obtain corresponding transformation $s_i, t_j$'s as obtained from the set of equations \ref{eqn:s1}, \ref{eqn:t1},\ref{eqn:s2} and \ref{eqn:t2}, which makes the marginals equal to $\frac{1}{2}$.
 	
 	Now, consider the transformed table with marginals equal to $0$. Note that, there are two possible cases.
 	\begin{enumerate}[label=(\alph*),leftmargin=4\parindent]
 		\item[\textbf{Case 1:}] Assume, $x \geq \frac{1}{4}$. In this case, $\max{x, (\frac{1}{2} - x)} = x$. Hence, from the formula of Kruskal's and Goodman's $\lambda_{C\mid R}$, we obtain;
 		\begin{align*}
 			\lambda_{C\mid R} & = \frac{p_{1m} + p_{2m} - p_{\cdot m}}{1 - p_{\cdot m}}\\
 			& = \frac{x + x - 0.5}{1 - 0.5}\\
 			& = (4x - 1)\\
 			& = \frac{2\sqrt{\theta}}{(1 + \sqrt{\theta})} - 1\\
 			& = \frac{\sqrt{\theta} - 1}{(1 + \sqrt{\theta})}
 		\end{align*}
 		Since, $x \geq \frac{1}{4}$, we have $(4x - 1) \geq 0$, and hence $\theta \geq 1$. Which shows that the above quantity is non-negative. 
 		Also note that, if we consider $\lambda_{R \mid C}$, then we know that $p_{m1} = p_{m2} = x$, and the column marginals will be $\frac{1}{2}$ as before. Therefore, similarly proceeding, we would obtain;
 		$$\lambda_{R \mid C} = \frac{\sqrt{\theta} - 1}{(1 + \sqrt{\theta})}$$
 		
 		\item[\textbf{Case 2:}] Assume, $x < \frac{1}{4}$. In this case, $\max{x, (\frac{1}{2} - x)} = (0.5 - x)$. Hence, from the formula of Kruskal's and Goodman's $\lambda_{C\mid R}$, we obtain;
 		\begin{align*}
 		\lambda_{C\mid R} & = \frac{p_{1m} + p_{2m} - p_{\cdot m}}{1 - p_{\cdot m}}\\
 		& = \frac{(0.5 - x) + (0.5 - x) - 0.5}{1 - 0.5}\\
 		& = (1 - 4x)\\
 		& = 1 - \frac{2\sqrt{\theta}}{(1 + \sqrt{\theta})}\\
 		& = \frac{1 - \sqrt{\theta}}{(1 + \sqrt{\theta})}
 		\end{align*}
 		Since, $x < \frac{1}{4}$, we have $(1 - 4x) \geq 0$, and hence $\theta < 1$. Which shows that the above quantity is non-negative. 
 		Also note that, if we consider $\lambda_{R \mid C}$, then we know that $p_{m1} = p_{m2} = x$, and the column marginals will be $\frac{1}{2}$ as before. Therefore, similarly proceeding, we would obtain;
 		$$\lambda_{R \mid C} = \frac{1 - \sqrt{\theta}}{(1 + \sqrt{\theta})}$$
 		
 		Combining results obtained from both the cases above, we get;
 		$$\lambda_{C\mid R} = \lambda_{R \mid C} = \left| \frac{\sqrt{\theta} - 1}{\sqrt{\theta} + 1} \right| = \vert Y \vert$$
 	\end{enumerate}
 	
\end{solution}
\vskip3pt
%%%%%%%%%%%%%%%%%%%%%%%%%%%%%%%%%%%%%%%%%%%%%%%%%%
% Problem 2
%%%%%%%%%%%%%%%%%%%%%%%%%%%%%%%%%%%%%%%%%%%%%%%%%%
\begin{problem}
\textbf{Minimum and maximum agreement} \hfill [12]\vskip3pt\noindent
% Problem statement
\begin{enumerate}
    \item[(a)] Consider a $2 \times 2$ contingency table. Given the marginals $p_{1\bigcdot}, p_{\bigcdot 1}$, compute the maximum and the minimum values of the agreement $p_{11} + p_{22}$, and hence compute the minimum and the maximum values of Cohen's $\kappa$. What are the minimizing and maximizing configurations?

    \item[(b)] Consider the measure
        \[
            \lambda_r = \frac{\sum_{i}p_{ii} - \frac{1}{2}(p_{m\bigcdot} + p_{\bigcdot m})}{1 - \frac{1}{2}(p_{m\bigcdot} + p_{\bigcdot m})},
        \]
    where $p_{m\bigcdot} + p_{\bigcdot m} = \max_{i} (p_{i\bigcdot} + p_{\bigcdot i})$. Give a decision-theoretic interpretation of this measure. What are its minimum and maximum values (marginals need not be fixed)?

    \item[(c)] Compute $\kappa$ and $\lambda_r$ for the data in Table~\ref{tab:emr}. Also compute the maximum and minimum values of these indices given the marginals.
    \begin{table}[!htbp]
    \centering
    \pgfplotstableread{
        Self   Yes     No
        Yes    4.5   10.6   
        No    11.2   73.7       
    }\chisquaredata

    \createcontingencytable{\chisquaredata}{Self}{Total}{Total}
    
    \pgfplotstabletypeset[
      every head row/.style={%
        before row={\toprule 
            & \multicolumn{3}{c}{EMR}\\            \cmidrule{2-4}},
        after row=\midrule},
      every last row/.style={after row=\bottomrule},
      columns/Self/.style={string type},
      columns={Self, Yes, No, {Total}},
    ]\contingencytable
    \caption{Self-report vs. electronic medical record (EMR) about receiving a prescription (in percentages).}
    \label{tab:emr}
    \end{table}
    \item[(d)] For an $I \times I$ table, formulate the tasks of finding the extremal values of $\sum_{i}p_{ii}$ when the marginals are fixed as linear programs. Use an LP solver to compute, given the marginals, the extremal values of $\kappa$ and $\lambda_r$ for the \texttt{SexualFun} data in the \textbf{R} package \texttt{vcd}. Compare the actual values of these measures against the extremal values.
\end{enumerate}
\end{problem}
%%%%%%%%%%%%%%%%%%%%%%%%%%%%%%%%%%%%%%%%%%%%%%%%%%
% Solution to problem 2
%%%%%%%%%%%%%%%%%%%%%%%%%%%%%%%%%%%%%%%%%%%%%%%%%%
\begin{solution*}
\begin{enumerate}
	\item[(a)] Given the value of the marginals $p_{\cdot 1}$ and $p_{1\cdot}$, we construct the following contingency table;
	\begin{table}[H]
		\centering 
		\begin{tabular}{cccc}
			\toprule
			& $C=1$ & $C=2$ & Total\\
			\midrule 
			$R = 1$ & $p_{11}$ & $p_{1\cdot} - p_{11}$ & $p_{1\cdot}$\\
			$R=2$ & $p_{\cdot 1} - p_{11}$ & $1 + p_{11} - p_{\cdot 1} - p_{1\cdot}$ & $1 - p_{1\cdot}$\\
			\midrule
			Total & $p_{\cdot 1}$ & $1 - p_{\cdot 1}$ & $1$\\
			\bottomrule
		\end{tabular}	
	\end{table}
	Therefore, we have $p_{22} = 1 + p_{11} - p_{\cdot 1} - p_{1\cdot}$. Hence, $p_{11} + p_{22} = 1 + 2p_{11} - p_{\cdot 1} - p_{1\cdot}$. Now, note that;
	$$\max\left\{0, p_{\cdot 1} +  p_{1\cdot} - 1\right\} \leq p_{11} \leq \min\left\{p_{\cdot 1}, p_{1\cdot}\right\}$$
	Therefore, we obtain;
	\begin{itemize}
		\item Maximum value of $p_{11} + p_{22}$ is $1 + 2\min\left\{p_{\cdot 1}, p_{1\cdot}\right\} - p_{\cdot 1} - p_{1\cdot}$.
		\item Minimum value of $p_{11} + p_{22}$ is $1 + 2\max\left\{0, p_{\cdot 1} +  p_{1\cdot} - 1\right\} - p_{\cdot 1} - p_{1\cdot}$.
	\end{itemize}
	
	Now, note that the formula for Cohen's Kappa is given by;
	$$\kappa = \dfrac{p_{11} + p_{22} - p_{1\cdot}p_{\cdot 1} - p_{2\cdot}p_{\cdot 2}}{1 - p_{1\cdot}p_{\cdot 1} - p_{2\cdot}p_{\cdot 2} }$$
	
	Without loss of generality, assume $p_{\cdot 1} < p_{1\cdot}$. Then, the maximizing configuration would look like;
	\begin{table}[H]
		\centering 
		\begin{tabular}{cccc}
			\toprule
			& $C=1$ & $C=2$ & Total\\
			\midrule 
			$R = 1$ & $p_{\cdot 1}$ & $p_{1\cdot} - p_{\cdot 1}$ & $p_{1\cdot}$\\
			$R=2$ & $0$ & $1 - p_{1\cdot}$ & $1 - p_{1\cdot}$\\
			\midrule
			Total & $p_{\cdot 1}$ & $1 - p_{\cdot 1}$ & $1$\\
			\bottomrule
		\end{tabular}	
	\end{table}
	In such table, the value of Cohen's Kappa would be;
	\begin{align*}
		\kappa & = \dfrac{p_{\cdot 1} + 1 - p_{1\cdot} - p_{1\cdot}p_{\cdot 1} - p_{2\cdot}p_{\cdot 2}}{1 - p_{1\cdot}p_{\cdot 1} - p_{2\cdot}p_{\cdot 2} }\\
		& = 1  - \dfrac{\vert p_{\cdot 1} - p_{1\cdot}\vert }{ 1 - p_{1\cdot}p_{\cdot 1} - p_{2\cdot}p_{\cdot 2} }
	\end{align*}
	On the other hand, we consider two cases for the minimizing configuration;
	\begin{enumerate}
		\item[\textbf{Case 1:}] Suppose, $p_{\cdot 1} +  p_{1\cdot} < 1$. In this case, the minimizing configuration would look like;
		\begin{table}[H]
			\centering 
			\begin{tabular}{cccc}
				\toprule
				& $C=1$ & $C=2$ & Total\\
				\midrule 
				$R = 1$ & $0$ & $p_{1\cdot}$ & $p_{1\cdot}$\\
				$R=2$ & $p_{\cdot 1}$ & $1 - p_{1\cdot} - p_{\cdot 1}$ & $1 - p_{1\cdot}$\\
				\midrule
				Total & $p_{\cdot 1}$ & $1 - p_{\cdot 1}$ & $1$\\
				\bottomrule
			\end{tabular}	
		\end{table}
		In such table, the value of Cohen's Kappa would be;
		\begin{align*}
		\kappa & = \dfrac{0 + 1 - p_{1\cdot} - p_{\cdot 1} - p_{1\cdot}p_{\cdot 1} - p_{2\cdot}p_{\cdot 2}}{1 - p_{1\cdot}p_{\cdot 1} - p_{2\cdot}p_{\cdot 2} }\\
		& = \dfrac{1 - p_{1\cdot} - p_{\cdot 1} - p_{1\cdot}p_{\cdot 1} - (1 - p_{1\cdot})( 1- p_{\cdot 1})}{1 - p_{1\cdot}p_{\cdot 1} - p_{2\cdot}p_{\cdot 2} }\\
		& = \dfrac{1 - p_{1\cdot} - p_{\cdot 1} - p_{1\cdot}p_{\cdot 1} - 1 + p_{1\cdot} + p_{\cdot 1} - p_{1\cdot}p_{\cdot 1}}{1 - p_{1\cdot}p_{\cdot 1} - p_{2\cdot}p_{\cdot 2} }\\
		& = \dfrac{- 2p_{1\cdot}p_{\cdot 1} }{1 - p_{1\cdot}p_{\cdot 1} - p_{2\cdot}p_{\cdot 2} }
		\end{align*}
		
		\item[\textbf{Case 2:}] Suppose, $p_{\cdot 1} +  p_{1\cdot} > 1$. In this case, the minimizing configuration would look like;
		\begin{table}[H]
			\centering 
			\begin{tabular}{cccc}
				\toprule
				& $C=1$ & $C=2$ & Total\\
				\midrule 
				$R = 1$ & $p_{1\cdot} + p_{\cdot 1} - 1$ & $1 - p_{\cdot 1}$ & $p_{1\cdot}$\\
				$R=2$ & $1 - p_{1\cdot }$ & $0$ & $1 - p_{1\cdot}$\\
				\midrule
				Total & $p_{\cdot 1}$ & $1 - p_{\cdot 1}$ & $1$\\
				\bottomrule
			\end{tabular}	
		\end{table}
		In such table, the value of Cohen's Kappa would be;
		\begin{align*}
		\kappa & = \dfrac{p_{1\cdot} + p_{\cdot 1} - 1 + 0 - p_{1\cdot}p_{\cdot 1} - p_{2\cdot}p_{\cdot 2}}{1 - p_{1\cdot}p_{\cdot 1} - p_{2\cdot}p_{\cdot 2} }\\
		& = \dfrac{-(1 - p_{1\cdot})(1 - p_{\cdot 1}) - p_{2\cdot}p_{\cdot 2}}{1 - p_{1\cdot}p_{\cdot 1} - p_{2\cdot}p_{\cdot 2} }\\
		& = \dfrac{ - 2p_{2\cdot}p_{\cdot 2}}{1 - p_{1\cdot}p_{\cdot 1} - p_{2\cdot}p_{\cdot 2} }
		\end{align*}
	\end{enumerate}
	
	
	\item[(b)] Consider the following game. There are two friends P and Q, who want to watch a movie in one of the $I$ movie theaters together. However, we consider the following two situations:
	\begin{enumerate}
		\item[(i)] They do not know where the other person is going.
		\item[(ii)] Each of them know where the other person is going.
	\end{enumerate}
	Now, consider the following loss function;
	$$L(x, \hat{x}) = (-1)\textbf{1}_{x = \hat{x}} + \textbf{1}_{x \neq \hat{x}}$$
	where $\textbf{1}_A$ denotes the indicator function of the event $A$. Note that, the above loss indicates that both person is unhappy if they go to different movie theaters while they both are happy if they go to same theater. 
	
	Note that, the expected loss (or risk) in the first situation is given by;
	\begin{align*}
		E_{(i)}\left( \text{Loss} \right) & = P\left( \hat{R} \neq \hat{C}\right) - P\left( \hat{R} = \hat{C} \right)\\
		& = \min_i \left( p_{ii} - p_{i\cdot} - p_{\cdot i} + 1  - p_{ii}\right)\\
		& = \min_i \left( - p_{i\cdot} - p_{\cdot i} + 1\right)\\
		& = 1 - p_{m\cdot} - p_{\cdot m}
	\end{align*}
	The above implication is true since both persons would try to minimize the expected loss given that the other person is also thinking the similar strategy.
	
	On the other hand, the expected loss (or risk) in the second situation is given by;
	\begin{align*}
	E_{(ii)}\left( \text{Loss} \right) & = \sum_{i}P\left( \hat{R} \neq \hat{C} \vert \hat{C} = i \text{ or } \hat{R} = i\right) - \sum_{i}P\left( \hat{R} = \hat{C} \vert \hat{C} = i \text{ or } \hat{R} = i \right)\\
	& = (1 - \sum_{i}p_{ii}) - \sum_{i}p_{ii}\\
	& = (1 - 2\sum_{i}p_{ii})
	\end{align*}
	
	Therefore, we consider the relative decrease in error due to newly added information;
	\begin{align*}
		\text{Relative decrease in Risk } & = \dfrac{E_{(i)}\left( \text{Loss} \right) - E_{(ii)}\left( \text{Loss} \right)}{E_{(i)}\left( \text{Loss} \right)}\\
		& = \dfrac{\left( 1 - p_{m\cdot} - p_{\cdot m} \right) - \left( 1 - 2\sum_{i}p_{ii} \right)}{\left( 1 - p_{m\cdot} - p_{\cdot m} \right)}\\
		& = \dfrac{2\sum_{i}p_{ii} - p_{m\cdot} - p_{\cdot m}}{\left( 1 - p_{m\cdot} - p_{\cdot m} \right)}
	\end{align*}
	The denominator is then changed accordingly to normalize the above quantity between $(-1)$ and $1$ which produces the measure $\lambda_r$.
	
	
	Observe that, if the marginals are not given then clearly, $\sum_{i}p_{ii} \leq 1$. Therefore, 
	$$\lambda_r = \frac{\sum_{i}p_{ii} - \frac{1}{2}(p_{m\bigcdot} + p_{\bigcdot m})}{1 - \frac{1}{2}(p_{m\bigcdot} + p_{\bigcdot m})}\leq \frac{1 - \frac{1}{2}(p_{m\bigcdot} + p_{\bigcdot m})}{1 - \frac{1}{2}(p_{m\bigcdot} + p_{\bigcdot m})} = 1$$
	
	To see that such maximizing configuration can be achieved, consider the following contingency table of probabilities;
	
	\begin{table}[H]
		\centering 
		\begin{tabular}{cccc}
			\toprule
			 & $C = 1$ & $C = 2$ & Total\\
			\midrule 
			$R = 1$ & $0.2$ & $0$ & $0.2$\\
			$R = 2$ & $0$ & $0.8$ & $0.8$\\
			Total & $0.2$ & $0.8$ & $1$\\
			\bottomrule
		\end{tabular}	
	\end{table}
	
	On the other hand, consider the probability;
	$$P\left[ R\neq i, C \neq i \right]=1-p_{i\cdot} -p_{\cdot i} + p_{ii} > 0$$
	
	From this, we obtain;
	$$\sum_{i} p_{ii} \geq \max_{i}p_{ii} \geq \max_{i}\left( p_{i\cdot} +p_{\cdot i} -1 \right) = \left( p_{m\cdot} +p_{\cdot m} -1 \right)$$
	
	Therefore, we have;
	\begin{align*}
		& \sum_{i} p_{ii} \geq \left( p_{m\cdot} +p_{\cdot m} -1 \right)\\
		\Rightarrow \quad & \sum_{i} p_{ii} - \frac{1}{2}\left( p_{m\cdot} +p_{\cdot m} \right) \geq \frac{1}{2}\left( p_{m\cdot} +p_{\cdot m} \right) -1\\
		\Rightarrow \quad & \frac{\sum_{i} p_{ii} - \frac{1}{2}\left( p_{m\cdot} +p_{\cdot m} \right)}{1 - \frac{1}{2}\left( p_{m\cdot} +p_{\cdot m} \right)} \geq (-1)\\
		\Rightarrow \quad & \lambda_r \geq (-1)
	\end{align*}
	
	To see that $(-1)$ can be achieved, consider the following contingency table;
	\begin{table}[H]
		\centering 
		\begin{tabular}{cccc}
			\toprule
			& $C = 1$ & $C = 2$ & Total\\
			\midrule 
			$R = 1$ & $0$ & $0.5$ & $0.5$\\
			$R = 2$ & $0.5$ & $0$ & $0.5$\\
			Total & $0.5$ & $0.5$ & $1$\\
			\bottomrule
		\end{tabular}	
	\end{table}
	
	
	\item[(c)] From the given table, we compute the contingency table comprising the estimated probabilities as follows;
	\begin{table}[H]
		\centering 
		\begin{tabular}{cccc}
			\toprule
			Self & Yes & No & Total\\
			\midrule 
			Yes & 0.045 & 0.106 & 0.151\\
			No & 0.112 & 0.737 & 0.849\\
			Total & 0.157 & 0.843 & 1\\
			\bottomrule
		\end{tabular}	
	\end{table}
	
	Therefore, the value of Cohen's Kappa for this table is going to be;
	\begin{align*}
		\kappa & = \dfrac{p_{11} + p_{22} - p_{1\cdot}p_{\cdot 1} - p_{2\cdot}p_{\cdot 2}}{1 - p_{1\cdot}p_{\cdot 1} - p_{2\cdot}p_{\cdot 2} }\\
		& = \dfrac{0.045 + 0.737 - (0.151\times 0.157) - (0.849\times 0.843)}{1 - (0.151\times 0.157) - (0.849\times 0.843) }\\
		& = \dfrac{0.782 - 0.739414}{1 - 0.739414}\\
		& = 0.163424
	\end{align*}
	
	Given the marginals, the maximizing value would be (as indicated in solution to problem 2(a)); 
	\begin{align*}
		\kappa_{\max} & = 1  - \dfrac{\vert p_{\cdot 1} - p_{1\cdot}\vert }{ 1 - p_{1\cdot}p_{\cdot 1} - p_{2\cdot}p_{\cdot 2} }\\
		& = 1 - \dfrac{\vert 0.151 - 0.157\vert}{1 - (0.151\times 0.157) - (0.849\times 0.843) }\\
		& = 0.976975
	\end{align*}
	
	Since, here $p_{\cdot 1} + p_{1 \cdot}$ is less than $1$, the minimizing value would be;
	
	\begin{align*}
	\kappa_{\min} & = 1  - \dfrac{-2p_{\cdot 1}p_{1\cdot}}{ 1 - p_{1\cdot}p_{\cdot 1} - p_{2\cdot}p_{\cdot 2} }\\
	& = 1 - \dfrac{(-2) 0.151 \times 0.157}{1 - (0.151\times 0.157) - (0.849\times 0.843)}\\
	& = -0.1819514
	\end{align*}
	
	Now, the value of $\lambda_r$ for the given contingency table would be;
	\begin{align*}
		\lambda_{r} & = \dfrac{p_{11} + p_{22} - \frac{1}{2}(p_{m\cdot} + p_{\cdot m})}{1 - \frac{1}{2}(p_{m\cdot} + p_{\cdot m})}\\
		& = \dfrac{0.045 + 0.737 - \frac{1}{2}(0.849 + 0.843)}{1 - \frac{1}{2}(0.849 + 0.843)}\\
		& = \dfrac{0.782 - 0.846}{1 - 0.846}\\
		& = -0.4155844
	\end{align*}
	
	Now, from the solution of problem 2(a), we also find that; maximum value of $p_{11} + p_{22}$ would be $1 + (2\times 0.151) - 0.151 - 0.157$, as $\min\left\{ p_{1\cdot}, p_{\cdot 1} \right\} = 0.151$, for the given marginals. Therefore, maximum value of $\lambda_r$ is;
	
	\begin{align*}
	\lambda_{r, \max} & = \dfrac{1 + (2\times 0.151) - 0.151 - 0.157 - \frac{1}{2}(0.849 + 0.843)}{1 - \frac{1}{2}(0.849 + 0.843)}\\
	& = \dfrac{0.994 - 0.846}{1 - 0.846}\\
	& = 0.961039
	\end{align*}
	
	and the minimum value of $p_{11} + p_{22}$ would be $(1 - 0.151 - 0.157) = 0.692$. Therefore, the minimum value of $\lambda_r$ given the marginals are going to be;
	\begin{align*}
	\lambda_{r, \min} & = \dfrac{0.692 - \frac{1}{2}(0.849 + 0.843)}{1 - \frac{1}{2}(0.849 + 0.843)}\\
	& = \dfrac{0.692 - 0.846}{1 - 0.846}\\
	& = -1
	\end{align*}
	
	\item[(d)] The problem of finding extremal values of $\sum_{i}p_{ii}$ can be rewritten as a linear programming problem with $I^2$ many variables of interest as follows;
	\begin{align*}
		\max \text{ or } \min & \sum_{i}p_{ii}\\
		\text{subject to the constraints;} &\\
		\sum_{j} p_{ij} & = p_{i\cdot} \qquad \forall i = 1, 2, \dots I\\ 
		\sum_{i} p_{ij} & = p_{\cdot j} \qquad \forall j = 1, 2, \dots I 
	\end{align*}

	To solve the linear programming, we first load the required packages in \emph{R}.
	\begin{Shaded}
			\KeywordTok{library}\NormalTok{(vcd)}\\
			\KeywordTok{library}\NormalTok{(lpSolve)}			
	\end{Shaded}

	Now, we load the \emph{SexualFun} data from \textbf{vcd} package.
	
	\begin{Shaded}
			\NormalTok{data \textless- } \StringTok{}\NormalTok{vcd}\OperatorTok{::}\NormalTok{SexualFun}\\
			\KeywordTok{ftable}\NormalTok{(data)}
	\end{Shaded}
	
	\begin{verbatim}
	Wife Never Fun Fairly Often Very Often Always fun
	Husband                                                       
	Never Fun                 7            7          2          3
	Fairly Often              2            8          3          7
	Very Often                1            5          4          9
	Always fun                2            8          9         14
	\end{verbatim}
	
	We first obtain the actual values of Cohen's Kappa \(\kappa\) and the
	measure \(\lambda_r\) for the actual table;
	
	\begin{Shaded}
			\NormalTok{kp \textless- }\StringTok{ }\KeywordTok{Kappa}\NormalTok{(data)   }\CommentTok{\#compute the actual Kappa }\\
			\KeywordTok{print}\NormalTok{(kp)}
	\end{Shaded}
	
	\begin{verbatim}
	value     ASE     z Pr(>|z|)
	Unweighted 0.1293 0.06860 1.885 0.059387
	Weighted   0.2374 0.07832 3.031 0.002437
	\end{verbatim}
	
	We note that the actual value of Cohen's Kappa for \emph{SexualFun} data
	is \(0.12933025\).
	
	Consider the following code which computes \(\lambda_r\);
	
	\begin{Shaded}
			\NormalTok{n \textless- }\StringTok{ }\KeywordTok{sum}\NormalTok{(data)}\\
			\NormalTok{rowMar \textless- }\StringTok{ }\KeywordTok{rowSums}\NormalTok{(data)}\OperatorTok{/}\NormalTok{n}\\
			\NormalTok{colMar \textless- }\StringTok{ }\KeywordTok{colSums}\NormalTok{(data)}\OperatorTok{/}\NormalTok{n}\\
			\NormalTok{a \textless- }\StringTok{ }\KeywordTok{sum}\NormalTok{(}\KeywordTok{diag}\NormalTok{(data))}\OperatorTok{/}\NormalTok{n}\\
			\NormalTok{b \textless- }\StringTok{ }\KeywordTok{max}\NormalTok{((rowMar }\OperatorTok{+}\StringTok{ }\NormalTok{colMar)}\OperatorTok{/}\DecValTok{2}\NormalTok{)}\\
			\NormalTok{lambda \textless-}\StringTok{ }\NormalTok{(a }\OperatorTok{-}\StringTok{ }\NormalTok{b)}\OperatorTok{/}\NormalTok{(}\DecValTok{1}\OperatorTok{-}\NormalTok{b)}\\
			\KeywordTok{print}\NormalTok{(lambda)}
	\end{Shaded}
	
	\begin{verbatim}
	[1] 0
	\end{verbatim}
	
	We find that the actual value of \(\lambda_r\) for \emph{SexualFun} data
	is \(0\).
	
	Now, we consider the minimization and maximization problem, where we try
	to find extremal values of \(\sum_i p_{ii}\) or correspondingly
	\(\sum_i n_{ii}\). We shall use \textbf{lpSolve} package to solve the
	corresponding linear programming for us.
	
	\begin{Shaded}
			\NormalTok{lpMax \textless- }\StringTok{ }\KeywordTok{lp.transport}\NormalTok{(}\DataTypeTok{cost.mat =} \KeywordTok{diag}\NormalTok{(}\DecValTok{4}\NormalTok{), }\DataTypeTok{direction =} \StringTok{"max"}\NormalTok{, }
			\DataTypeTok{row.signs =} \KeywordTok{rep}\NormalTok{(}\StringTok{"=="}\NormalTok{, }\DecValTok{4}\NormalTok{), }\DataTypeTok{row.rhs =} \KeywordTok{rowSums}\NormalTok{(data), }
			\DataTypeTok{col.signs =} \KeywordTok{rep}\NormalTok{(}\StringTok{"=="}\NormalTok{, }\DecValTok{4}\NormalTok{), }\DataTypeTok{col.rhs =} \KeywordTok{colSums}\NormalTok{(data))}
	\end{Shaded}
	
	The maximizing configuration would look like;
	
	\begin{Shaded}
			\KeywordTok{print}\NormalTok{(lpMax}\OperatorTok{\$}\NormalTok{solution)}
	\end{Shaded}
			
	\begin{verbatim}
			[,1] [,2] [,3] [,4]
			[1,]   12    7    0    0
			[2,]    0   20    0    0
			[3,]    0    1   18    0
			[4,]    0    0    0   33
	\end{verbatim}
			
	Therefore, the maximum value of Cohen's kappa and \(\lambda_r\) is
	obtained using the following code;
	
	\begin{Shaded}
			\KeywordTok{Kappa}\NormalTok{(lpMax}\OperatorTok{\$}\NormalTok{solution)}
	\end{Shaded}
			
	\begin{verbatim}
			value     ASE     z   Pr(>|z|)
			Unweighted 0.8799 0.03969 22.17 6.661e-109
			Weighted   0.9291 0.02348 39.57  0.000e+00
	\end{verbatim}
			
	\begin{Shaded}
			\NormalTok{a \textless- }\StringTok{ }\NormalTok{lpMax}\OperatorTok{\$}\NormalTok{objval}\OperatorTok{/}\NormalTok{n}\\
			\NormalTok{lambda \textless- }\StringTok{ }\NormalTok{(a }\OperatorTok{-}\StringTok{ }\NormalTok{b)}\OperatorTok{/}\NormalTok{(}\DecValTok{1}\OperatorTok{-}\NormalTok{b)}\\
			\KeywordTok{print}\NormalTok{(lambda)}
	\end{Shaded}
	
	\begin{verbatim}
	[1] 0.862069
	\end{verbatim}
	
	Therefore, the maximum value of Cohen's Kappa given the marginals is
	\(0.8799\), while the maximum value of \(\lambda_r\) given the marginals
	is \(0.8621\).
	
	We use similar method to find the minimum value of these measures given
	the marginals.
	
	\begin{Shaded}
			\NormalTok{lpMin \textless- }\StringTok{ }\KeywordTok{lp.transport}\NormalTok{(}\DataTypeTok{cost.mat =} \KeywordTok{diag}\NormalTok{(}\DecValTok{4}\NormalTok{), }\DataTypeTok{direction =} \StringTok{"min"}\NormalTok{, }
			\DataTypeTok{row.signs =} \KeywordTok{rep}\NormalTok{(}\StringTok{"=="}\NormalTok{, }\DecValTok{4}\NormalTok{), }\DataTypeTok{row.rhs =} \KeywordTok{rowSums}\NormalTok{(data), }
			\DataTypeTok{col.signs =} \KeywordTok{rep}\NormalTok{(}\StringTok{"=="}\NormalTok{, }\DecValTok{4}\NormalTok{), }\DataTypeTok{col.rhs =} \KeywordTok{colSums}\NormalTok{(data))}
	\end{Shaded}
	
	The minimizing configuration would look like;
	
	\begin{Shaded}
			\KeywordTok{print}\NormalTok{(lpMin}\OperatorTok{\$}\NormalTok{solution)}
	\end{Shaded}
			
			\begin{verbatim}
			[,1] [,2] [,3] [,4]
			[1,]    0    0    0   19
			[2,]    0    0    6   14
			[3,]    0   19    0    0
			[4,]   12    9   12    0
			\end{verbatim}
			
		\begin{Shaded}
			\KeywordTok{Kappa}\NormalTok{(lpMin}\OperatorTok{\$}\NormalTok{solution)}
	\end{Shaded}
	
	\begin{verbatim}
	value     ASE      z   Pr(>|z|)
	Unweighted -0.3661 0.01593 -22.98 7.992e-117
	Weighted   -0.5607 0.02691 -20.84  1.951e-96
	\end{verbatim}
	
	\begin{Shaded}
			\NormalTok{a \textless- }\StringTok{ }\NormalTok{lpMin}\OperatorTok{\$}\NormalTok{objval}\OperatorTok{/}\NormalTok{n}\\
			\NormalTok{lambda \textless- }\StringTok{ }\NormalTok{(a }\OperatorTok{-}\StringTok{ }\NormalTok{b)}\OperatorTok{/}\NormalTok{(}\DecValTok{1}\OperatorTok{-}\NormalTok{b)}\\
			\KeywordTok{print}\NormalTok{(lambda)}
	\end{Shaded}
			
	\begin{verbatim}
			[1] -0.5689655
	\end{verbatim}
			
	Therefore, the maximum value of Cohen's Kappa given the marginals is \(-0.3661\), while the maximum value of \(\lambda_r\) given the marginals is \(-0.5689\).
			
	
\end{enumerate}
\end{solution*}

\section*{Acknowledgments}
\qquad I sincerely want to extend my thanks to Tamojit Sadhukhan who helped me with some of the solutions.


\end{document}






